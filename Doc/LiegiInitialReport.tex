\documentclass[10pt]{article}

%% FOR THE MISSING FIGURES !!!
%%  \image{missing}{LABEL,CHANGE ME}{\td{missing figure explanation}}{0.8}

\usepackage[paperwidth=21cm, paperheight=29.7cm, tmargin=2cm, bmargin=2cm, outer=2cm, inner=2cm]{geometry}

%  colors
\usepackage{color}
\definecolor{deepblue}{rgb}{0,0,0.5}
\definecolor{deepred}{rgb}{0.6,0,0}
\definecolor{deepgreen}{rgb}{0,0.5,0}
\definecolor{deepgreen2}{rgb}{0,0.4,0}
\definecolor{violet}{rgb}{0,0.4,0.4}

\usepackage[utf8]{inputenc}
\usepackage{graphicx}
\usepackage{multicol}
\usepackage{todonotes}
\usepackage{float}
\usepackage{gensymb}
\usepackage{amsmath}
\usepackage{cite}
\usepackage{comment}
\usepackage{bbold}

% better frames around code
\usepackage[framemethod=TikZ]{mdframed}
\mdfdefinestyle{listingstyle}{
backgroundcolor=black!3,
outerlinewidth=0.25pt,
outerlinecolor=black,
innerleftmargin=5pt,
innerrightmargin=5pt,
innertopmargin=0pt,
innerbottommargin=0pt
}

% listings for python
\usepackage{listings}
\lstset{
language=Python, 
mathescape, 
numbers=left, 
numberstyle=\tiny, 
basicstyle=\scriptsize, 
showstringspaces=false, 
breaklines=true,
keywordstyle=\color{deepblue}, 
emphstyle=\color{deepred}, 
stringstyle=\color{deepgreen}, 
morecomment=[s][\color{violet}]{'''}{'''}, 
morecomment=[l][\color{deepgreen2}]{\#}}

% to load external code
\newcommand{\codeExt}[1]{\begin{mdframed}[style=listingstyle]\lstinputlisting{#1}\end{mdframed}}

%to make the ref clickable in the pdf
\usepackage[hidelinks]{hyperref}

%for the enumeration separation
\usepackage{enumitem}
\setlist{nosep} % or \setlist{noitemsep} to leave space around whole list

%image folder
\graphicspath{ {images/} }

\usepackage{bibentry}
\nobibliography*
%%%%%%%%%%%%%%%%%%%%%%%%%%%%%%%%%%%%%%%%%%%%%%

% better captions %%%%%%%%%%%%%%%%%%%%%%%%
\usepackage[labelfont={bf}, font={small,it}, margin=1cm]{caption}
%%%%%%%%%%%%%%%%%%%%%%%%%%%%%%%%%%%%%%%%%%

\newcommand{\bee}{\begin{enumerate}}
\newcommand{\ee}{\end{enumerate}}
\newcommand{\bi}{\begin{itemize}}
\newcommand{\ei}{\end{itemize}}
\newcommand{\td}[1]{\todo[inline]{#1}}
\newcommand{\image}[4]{\begin{figure}[h!]\centering \includegraphics[scale=#4]{#1} \caption{#3} \label{#2} \end{figure}}

% colored matrices %%%%%%%%%%%%%%%%%
\usepackage{tikz}
\usetikzlibrary{arrows,matrix,positioning}
%%%%%%%%%%%%%%%%%%%%%%%%%%%%%%%%%%%%

% better verbatims %%%%%%%%%%%%%%%%%
\usepackage{fancyvrb}
\RecustomVerbatimCommand{\VerbatimInput}{VerbatimInput}{fontsize=\footnotesize, frame=single, framesep=2em, numbers=left, baselinestretch=0.7}
%%%%%%%%%%%%%%%%%%%%%%%%%%%%%%%%%%%%

% better matrices %%%%%%%%%%%%%%%%%%%%%%%%%%%%%%%%%%%
\usepackage{mathtools}
\usepackage{xparse}
\usepackage{tikz}
\usetikzlibrary{matrix,backgrounds}
\pgfdeclarelayer{myback}
\pgfsetlayers{myback,background,main}

\tikzset{mycolor/.style = {line width=1bp,color=#1}}%
\tikzset{myfillcolor/.style = {draw,fill=#1}}%

\NewDocumentCommand{\highlight}{O{black!40} m m}{%
\draw[mycolor=#1] (#2.north west)rectangle (#3.south east);
}

\NewDocumentCommand{\fhighlight}{O{black!40} m m}{%
\draw[myfillcolor=#1] (#2.north west)rectangle (#3.south east);
}
%%%%%%%%%%%%%%%%%%%%%%%%%%%%%%%%%%%%%%%%%%%%%%%%%%%%%%

\title{Progresses}
\author{Alessio Valentini}

\begin{document}
\maketitle
%\tableofcontents
\listoftodos
\section{First things}
So, these articles are REALLY hard for me to grasp \cite{nikodem2016quantum,mignolet2014charge,nikodem2016controlling}.

\vspace{2cm}

\begin{tabular}{|c|l|}
\hline
Symbol & Meaning\\
\hline
s & Number of Electronic States\\
n & Number of Electrons\\
N & Number of Nuclei\\
$\hat{H}$ & total Hamiltonian\\
$\hat{H}^0$ & electronic Hamiltonian\\
\hline
\end{tabular}

\vspace{2cm}
\section{Fixed nuclei}
We will integrate the TDSE:
\begin{equation}\label{tdse}
i\hbar \dfrac{\partial\Psi}{\partial t} = \hat{H} \Psi
\end{equation}

The problem is to propagate the effect of a short electromagnetic pulse in the state distribution of LiH at equilibrium geometry. In fact, the pulse has the effect of mixing states depending on the transition dipole between them.

\subsection{Molcas calculation}
First I need to make a single point of LiH at equilibrium distance 1.63 \AA.
\bee
\item Basis: 6-31G**
\item CASSCF(4/16)
\item State Average 8 (lowest energy states)
\ee
\VerbatimInput{files/LiH.input}
From this calculation we need the state energies (Rasscf) and the transition dipole matrices (Rassi keyword mees).

\td{This setup gave me an equilibrium length of 2.12 \AA, way out of the 1.63 \AA \ reported in \cite{nikodem2016controlling}. This can be fault of the small basis set (small in terms of diffuse functions).}


\subsection{Equations}
We will consider the following total wavefunction:
\begin{equation}\label{boexp}
\Psi(t) = \sum^{s}_i c_i(t) |\Psi_i\rangle
\end{equation}
Where the sum is over the $s$ different electronic states $|\Psi_i\rangle$ solution of our Molcas calculation (this is the BO expansion). We are propagating the coefficients, so $c_i$ will change with time according to this equation:
\begin{equation}\label{ctdse}
\dot{c_i}=\dfrac{\partial c_i(t)}{\partial t} = \dfrac{1}{i\hbar} \sum_j \hat{H}_{ij} c_j(t)
\end{equation}
Where the Hamiltonian elements $H_{ij}$ are something like:
\begin{equation}\label{hami1}
H_{ij} = \langle i|\hat{H}^0 + \hat{T}_N + \hat{\mu}|j\rangle
\end{equation}
using the BO approximation (for $\hat{T}_N$) we could consider:
\begin{equation}\label{hami2}
H_{ij} = \langle i|\hat{H}^0|j\rangle \delta_{ij} + \langle i| \hat{T}_N |j\rangle \delta_{ij} - \vec{E}_0(t) \vec{\mu}_{ij}
\end{equation}

In the single point problem:
\bee
\item As we are not moving the geometry, there is no need of the Kinetic Energy operator on the nuclei, so $\hat{T}_N=0$.
\item $\langle i|\hat{H}^0|j\rangle \delta_{ij}$ can be seen as matrix of the solutions of Molcas calculation.\\
\begin{equation}
\begin{pmatrix*}
  V_1 & 0 & 0 \\
  0 & V_2 & 0 \\
  0 & 0 & V_3 \\
\end{pmatrix*}
\end{equation}
\item $\vec{E}_0(t)$ is the amplitude of our pulse (calculated as a function of time by using equation \ref{pulse}). This of course will be coded along the Cartesian directions so that $\vec{E}_0(t)=E_x(t)\hat{x}+E_y(t)\hat{y}+E_z(t)\hat{z}$. It is possible to decide the orientation of our pulse with respect to the molecule, and this is something that is achievable experimentally. Having a strongly polarized molecule like LiH, we can experimentally lock (align) them with respect to the lab reference frame, so that we can control the directionality with polarized pulses.
\item $\vec{\mu}_{ij}$ are the elements of the transition dipole matrix, obtained by Molcas. The higher the value of $\vec{\mu}_{ij}$, the higher the amount of amplitude that will be transferred from $c_i$ to $c_j$ (in case of a pulse). Anyway, this also depends on the pulse $\vec{E}_0(t)$, thus the scalar product with $\vec{\mu}_{ij}$. In fact, the whole term $- \vec{E}_0(t) \vec{\mu}_{ij}$ is the one changing the state amplitudes $c_i$.
\ee

\subsection{Electromagnetic pulse}

\bee
\item we intend a \textit{weak} pulse when it is around $5*10^{-4}$/$10^{-3}$ a.u. In this case we are basically linear, and by doubling the amplitude of the electric field, the transition will vary with a factor 4
\item in a \textit{strong} pulse, instead, $5*10^{-3}$/$10^{-2}$, we start to see non linear effects, such as multiple jumps between states that are resonant with our wavelength.
\ee

$\omega$ is more complicated and needs to be explained with the Fourier transform.

\begin{equation}\label{pulse}
\vec{E}(t) = \sum_d E_d cos(\omega t + \phi)\cdot e^{-\dfrac{(t-t_0)^2}{2\sigma^2}} \ \ \ \ \ \   d={x,y,z}
\end{equation}
\td{$\omega$ and Fourier transform}
where:
\bee
\item $\sigma$ is the standard deviation of the gaussian
\item $2\pi \nu = \omega = \dfrac{2\pi}{T}   \ \ \ \ \ \    T = period$
\item $E_x$ is the amplitude (component along x axis, fig.\ref{im:pulse}).
\item $t_0$ is the offset, where the gaussian is centered
\item $\phi$ is the initial phase of the wave.
\ee

\image{pulseh.png}{im:pulse}{the pulse shape in the x axis component $E_x(t)$ given by equation \ref{pulse}. This is done from \url{https://rechneronline.de/function-graphs/} using function \texttt{3sin(10x)*exp(-(x-2)\^{}2/3)}. This pulse of amplitude 3 and frequency 10 is gaussian shaped, centered in $x=2$ and with a $\sigma = 3$.}{0.4}

\subsection{Integration}
So we have to propagate with time eq. \ref{ctdse}. We will basically have a function that calculates our derivative at time $t$, starting from the value $c_i(t)$: $$\dot{c}_i(t)=\dfrac{\partial c_i(t)}{\partial t}=f(t,c_i(t)).$$ This function $f$, that calculates the changes on the coefficients $c_i$, is for us eq. \ref{ctdse}.
\subsubsection{Euler method}
This is the easiest integrator (a simple rectangle):
\begin{equation}
c_i (t + \Delta t) = c_i(t) + f(t,c_i(t)) \Delta t 
\end{equation}
meaning that from an initial value of $c_i(t)$, you can calculate the value of $c_i$ at time $(t + \Delta t)$, by adding the derivative $f(t,c_i(t))$ multiplied by $\Delta t$. Here, $f(t,c_i(t))$ stands for equation \ref{ctdse} applied having $c_i(t)$ as a value at time $t$. 
\subsubsection{Runge-Kutta 4th order}
This integration is more precise. Here we introduce 4 new coefficients $k$:
\begin{equation}
c (t + \Delta t) = c(t) + \dfrac{k_1+2k_2+2k_3+k_4}{6}
\end{equation}
where:
\begin{equation}
k_1=f(t,c_i(t)) \Delta t
\end{equation}
\begin{equation}
k_2=f\left(t + \dfrac{\Delta t}{2},c_i(t) + \dfrac{k_1}{2}\right) \Delta t
\end{equation}
\begin{equation}
k_3=f\left(t + \dfrac{\Delta t}{2},c_i(t) + \dfrac{k_2}{2}\right) \Delta t
\end{equation}
\begin{equation}
k_4=f\left(t + \Delta t,c_i(t) + k_3\right) \Delta t
\end{equation}
so that we need to recompute for 4 times the function $f$, each time at slightly different values of $t$ and $c_i$. This means that in the code routine the whole $c_i$ array must be propagated 4 times, before giving the definitive values at time $t+\Delta t$.

\subsubsection{Dipole Moment}
Along the integration we can calculate the dipole moment $\vec{\mu}(t)$. The wavefunction amplitudes, the coefficients of equation \ref{boexp}, are changing with time, and as a consequence the molecule dipole moment is also changing. To have the components of $\vec{\mu}(t)$ we should implement the following equation:
\begin{equation}
\vec{\mu}(t) = \sum_i\sum_j c^*_i(t)c_j(t)\vec{\mu}_{ij}
\end{equation}
The hermitian $\sum_i\sum_j c^*_i(t)c_j(t)$ matrix can be also seen as two triangular matrix plus the diagonal:
\begin{equation}
\sum_i\sum_j c^*_i(t)c_j(t) = \sum_i\sum_{j<i} c^*_i(t)c_j(t) + \sum_i\sum_{j<i} c^*_j(t)c_i(t) + \sum_i c^*_i(t)c_i(t)
\end{equation}
in the end:
\begin{equation}
\sum_i\sum_j c^*_i(t)c_j(t) = \sum_i\sum_{j<i} (c^*_i(t)c_j(t) + c^*_j(t)c_i(t)) + \sum_i c^*_i(t)c_i(t)
\end{equation}
This means that:
\begin{equation}
\sum_i\sum_j c^*_i(t)c_j(t) = 2 \left(\sum_i\sum_{j<i} c^*_j(t)c_i(t)\right) + \sum_i c^*_i(t)c_i(t)
\end{equation}
now, if the complex number $c_i=(a_i+ib_i)$ and $c^*_i=(a_i-ib_i)$:
\begin{equation}
c^*_i(t)c_j(t) = a_i a_j - ia_j b_i + ia_i b_j + b_i b_j =  (a_i a_j + b_i b_j) + i(a_i b_j - a_j b_i)
\end{equation}
this ends up being:
\begin{equation}
\sum_i\sum_j c^*_i(t)c_j(t) = \left(\sum_i\sum_{j<i} 2(a_i a_j + b_i b_j)\right) + \sum_i c^*_i(t)c_i(t)
\end{equation}
I still have to implement it like this (this actually saves half of the computations), but... right now I just implemented the main formula.

\td{go implementing this, what are you waiting??}


\section{Moving the Nuclei}
In order to move the nuclei we need to consider our system on a grid of different distances between Li and H. I will call these points $g$. My problem is now 1 rank higher, so my new object to propagate is a matrix:

\begin{equation}
\begin{pmatrix*}
 g_{11} & g_{21} & ... & g_{g1} \\
 g_{12} & g_{22} & ... & g_{g2} \\
 ... & ... & ... & ... \\
 g_{1i} & g_{2i} & ... & g_{gi}\\
\end{pmatrix*}
\end{equation}

Each of these elements are complex numbers, and the dimensionality is $g$:\textit{number of points in grid} and $i$:\textit{number of electronic states}. That is, in short, for each electronic state I have a 1D array of grid points, where I have calculated the potentials and the dipole moment matrices. In the following the coordinate $x$ is actually the x location of the hydrogen, but being the lithium in the origin, it is identical to consider the distance between two atoms.

\subsection{The initial state}
The initial state is gonna be a normalized gaussian that is solution of the harmonic oscillator, and that has the correct displacement and width along the grid for our system. The correct harmonic solution for any electronic state is the following:
\begin{equation}
\psi_n(x)=\dfrac{1}{(2^n n!)^{1/2}}\left(\dfrac{m\omega}{\pi\hbar}\right)^{1/4}exp\left(\dfrac{-m\omega x^2}{2\hbar}\right)H_n \left( \left( \dfrac{m\omega}{\hbar}\right)^{1/2}x \right),\ \ \ \ \ \ \ n=0,1,2,3, ...
\end{equation}
where $H_n$ are the hermite polynomials. In our problem, anyway, we want to start from the ground state ($n=0$) and we will normalize our vector just by calculating the array norm and dividing the array by it. This means that we drop the analytical calculation of the normalization constant, and even the Hermite polynomial part, as, when $n=0$: $$H_n \left( \left( \dfrac{m\omega}{\hbar}\right)^{1/2}x \right)=1$$
So, we need to calculate along $g_i$ the non-normalized gaussian:
\begin{equation}
\psi_0(x) = exp\left(-\dfrac{m\omega x^2}{2\hbar}\right)
\end{equation}
To be able to make $\hbar=1$ I need to be sure that both $m$ and $\omega=(k/m)$ are expressed in AU. In our problem this is done by considering:
\bee
\item contracted mass of the system:    
\begin{equation}\label{ciao}
m_{LiH} = \dfrac{m_H m_{Li}}{m_H + m_{Li}} 
\end{equation}

where  $m_H=1.00794*1836.15$ and $m_{Li}=6.941*1836.15$.
\item $\omega$, the oscillation frequency, depends on $m$ and on force constants $k$, i.e. second derivative of the energy along the x direction (where the grid is):
\begin{equation}
\omega=(k/m)^{1/2} \ \ \ \ \ \ \ where: \ \ k=\dfrac{\partial^2V}{\partial x^2}
\end{equation}
\item the energy are in Hartree and the coordinates of the array $g$ are expressed in Bohr.\\
\ee
This, when we are in atomic units, is how we calculate $\sigma$ to get the gaussian values along $g_i$:
\begin{equation}
exp\left(-\dfrac{(x-x_0)^2}{2\sigma^2}\right) = exp\left(-\dfrac{m\omega(x-x_0)^2}{2}\right)
\end{equation}
\begin{equation}
\dfrac{1}{\sigma^2} = m\omega \ \ \ \ \ \ \ \ \ \sigma = \left(\dfrac{1}{m\omega}\right)^{1/2}
\end{equation}
here $x_0$ is the distance where the ground state energy is in a minimum, and $\sigma$ will be calculated using the energies of the ground state as well (force constants) and the atom masses. We know from B.Mignolet that $\omega$ is 1606 $cm^{-1}$.

\subsection{Attempt at being rigorous}
In the continuous space, whatever state $|\chi\rangle$ can be projected in the representation of eigenfunctions of the position operator (along R) $\chi(R)$ (a complete basis):
\begin{equation}
|\chi\rangle = \mathbb{1} |\chi\rangle = \int dR | R \rangle \langle R | \chi \rangle = \int dR \chi(R) | R \rangle 
\end{equation}
It is possible to define a subspace of R composed by a discrete distribution along R (be aware of the resolution of such subspace) such as $|\theta_g\rangle$. Those are basically delta functions on R:
\begin{equation}
|\theta_g\rangle : \langle R | \theta_g \rangle = \delta(R-R_g)
\end{equation}
we can now project our state $|\chi\rangle$ in this discrete space $|\theta_g\rangle$:
\begin{equation}
|\chi\rangle = \mathbb{1} |\chi\rangle = \sum_g  |\theta_g\rangle \langle \theta_g | \chi \rangle = \sum_g \chi_g |\theta_g\rangle
\end{equation}
\image{wvandspaces}{wvandspaces}{two different representations of the vector $|\chi\rangle$, green is continuous and red is discrete}{0.3}

We do have the TDSE to solve now
\begin{equation}
i\hbar\dfrac{d|\Psi(t)\rangle}{dt} = \hat{H}|\Psi(t)\rangle
\end{equation}
but this time the wavefunction is projected into a subspace of internal coordinates on spacial grid points $\theta_g$ and on the electronic states $\Phi_i$:
\begin{equation}
|\Psi(t)\rangle = |\Phi_i,\theta_g\rangle
\end{equation}
\begin{equation}
|\Psi(t)\rangle = \sum_g \sum_i c_{ig}(t) |\Phi_i,\theta_g\rangle
\end{equation}
\image{wvandspacesMulti}{wvandspacesMulti}{The representation of $|\Psi\rangle$ into $\sum_g \sum_i c_{ig}(t) |\Phi_i,\theta_g\rangle$.}{0.3}
We can write down how every single coefficient $c_{ig}(t)$ changes with time:
\begin{equation}
\dot{c}_{ig}=\dfrac{\partial}{\partial t} c_{ig}(t) = 
\end{equation}

\subsection{Beyond BO}
Indexes are $[i,j]$ for electronic states and $[g,h]$ for nuclear coordinates along x axis.
\begin{multline}
         \langle \Psi | \hat{T}_{N}  | \Psi \rangle = 
         \sum_h \sum_j \langle \Phi_j,\theta_h | c^*_{jh}(t)  \hat{T}_{N}   \sum_g \sum_i c_{ig}(t) |\Phi_i,\theta_g \rangle =  
         \sum_{h,g} \sum_{j,i} \langle \Phi_j,\theta_h | c^*_{jh}(t)  \hat{T}_{N}  c_{ig}(t) |\Phi_i,\theta_g \rangle
\end{multline}

\begin{multline}
\langle \Psi | \hat{T}_{N}  | \Psi \rangle = \langle \Psi | \hat{T}_{N}  | \Psi \rangle
\end{multline}

%\begin{multline}
%\langle \theta_h | \langle \Phi_j | c^*_{jh}(t)  \hat{T}_{N}  c_{ig}(t)  |\Phi_i\rangle | \theta_g \rangle =
%-\dfrac{\hbar ^2}{2m} \langle \psi_{jh} | \dfrac{d^2}{dx^2}c_{ig}(t) | \psi_{ig} \rangle =
%-\dfrac{\hbar ^2}{2m} \langle \psi_{jh} | \dfrac{d}{dx} \left[ \dfrac{d}{dx} c_{ig}(t) | \psi_{ig} \rangle + c_{ig}(t) \dfrac{d}{dx}| \psi_{ig} \rangle \right] = \\
%-\dfrac{\hbar ^2}{2m} \langle \psi_{jh} | \left[ \dfrac{d^2}{dx^2} c_{ig}(t) | \psi_{ig} \rangle + \dfrac{d}{dx} c_{ig}(t) \dfrac{d}{dx}| \psi_{ig} \rangle + \dfrac{d}{dx} c_{ig}(t) \dfrac{d}{dx}| \psi_{ig} \rangle + c_{ig}(t) \dfrac{d^2}{dx^2}| \psi_{ig} \rangle \right] = \\
%-\dfrac{\hbar ^2}{2m} \left[ \langle \psi_{jh} | \dfrac{d^2}{dx^2} c_{ig}(t)          | \psi_{ig} \rangle 
%                         + 2 \langle \psi_{jh} | \dfrac{d}{dx} c_{ig}(t) \dfrac{d}{dx}| \psi_{ig} \rangle 
%                         +   \langle \psi_{jh} | c_{ig}(t) \dfrac{d^2}{dx^2}          | \psi_{ig} \rangle \right]
%\end{multline}

%\td{Stephan, why is here $| \psi_{t;x} \rangle$ and not $| \psi_{ig} \rangle$}


\subsection{About Scalar Product and integrals}
Just to let you know. In every Scalar product an "integral is hidden". Remember:
\begin{equation}
I = \int | x \rangle \langle x | dr \ \ \ \ \ \ \ \ and \ \ \ \ \ \ \ \langle x | y \rangle = \langle y | x \rangle^*
\end{equation}
Then:
\begin{equation}
\langle a(r) | b(r) \rangle = \langle a(r) | I | b(r) \rangle = \int dr \langle a(r) | r \rangle \langle r | b(r) \rangle = \int dr \langle r | a(r) \rangle^* \langle r | b(r) \rangle
\end{equation}
This leads to the common formula, as $\langle r | b(r) \rangle=b(r)$ and $\langle r | a(r) \rangle^*=a^*(r)$:
\begin{equation}
\int dr \ \ a^*(r) b(r)
\end{equation}
\vspace{10cm}
\begin{equation}
H_{ij} = \langle i|\hat{H}^0 + \hat{T}_N + \hat{\mu}|j\rangle
\end{equation}
\begin{equation}
H_{ij} = \langle i|\hat{H}^0|j\rangle \delta_{ij} + \langle i| \hat{T}_N |j\rangle \delta_{ij} - \vec{E}(t) \vec{\mu}_{ij}
\end{equation}
\begin{equation}
i\dfrac{dc_i(t)}{dt}=\sum_j H_{ij}c_j(t)
\end{equation}
\begin{equation}
H_{ij}=T_{ij}+V_{ij}\delta_{ij} - \vec{E} \cdot \vec{\mu}_{ij}
\end{equation}
\pagebreak
\section{Future work}
\subsection{Displaying the electronic density}
The idea is to create a grid around the molecule with the value of the electronic density. This can then be represented by an isosurface. Each electronic state has its own density, and by propagating the coefficients in equation \ref{boexp} we "populate different electronic states", basically reshaping the overall density.

\section{Questions}
\td{which units is $\hbar$}
\td{rassi energies or rasscf energies?}
\pagebreak
\subsection{Computer settings}
So at first I was struggling a little bit with Molcas output parsers. Then I decided to let it go and try hdf5 files.

I installed hdf5 (this works in opensuse):
\begin{verbatim}
zypper install hdf5
\end{verbatim}
and recompiled Molcas with this flag enabled.

Now the calculation is giving me new \texttt{\$project.\$module.h5} files in the working directory. I was able to modify Molcas code to get this matrix directly, without the need of log parsers.

\td{check the \href{https://cobalt.itc.univie.ac.at/molcasforum/}{Molcas forum}}

\subsubsection{Installing the ipython environment}
\begin{verbatim}
$ wget https://repo.continuum.io/miniconda/Miniconda3-latest-Linux-x86_64.sh
$ chmod 744 Miniconda3-latest-Linux-x86_64.sh
$ ./Miniconda3-latest-Linux-x86_64.sh
$ conda install numpy pandas h5py ipython ipykernel matplotlib scipy
$ pip install --upgrade pip
$ pip install pymonad pyprof2calltree line_profiler
$ pip3 install --upgrade pip
$ pip3 install jupyter mpld3
\end{verbatim}

I then changed the \texttt{.bashrc} file to lunch ipython3. I also wrote two small python routines that reads up the h5 file. I also installed this package \textit{kcachegrind}.
To profile, these are the commands:
\begin{verbatim}
$ python3 -m cProfile -o prof.out ./Main.py
$ pyprof2calltree -i prof.out -k
\end{verbatim}
To line-to-line profiling, I activate the decorators on the functions I am running:
\begin{verbatim}
@profile
def HamiltonianEle(Ici,Icj,matV,matMu,pulseV):
...
\end{verbatim}
then:
\begin{verbatim}
$ kernprof -l -v Main.py
\end{verbatim}
%\section{The Code}
\subsection{Main}
\codeExt{Code/Main.py}

\subsection{SinglePointIntegrator}
\codeExt{Code/SinglePointIntegrator.py}

\subsection{Molcas h5 reader}
\codeExt{Code/h5Reader.py}

\subsection{Integrator}
\codeExt{Code/Integrator.py}

\subsection{Pulse}
\codeExt{Code/Pulse.py}

\subsection{General Functions}
\codeExt{Code/GeneralFunctions.py}

\bibliographystyle{ChemEurJ}
\bibliography{biblio}


\end{document}